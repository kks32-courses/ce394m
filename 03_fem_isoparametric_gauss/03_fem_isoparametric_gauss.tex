%**************************************************************************************
% License:
% CC BY-NC-SA 4.0 (http://creativecommons.org/licenses/by-nc-sa/4.0/)
%**************************************************************************************

\documentclass[notes]{beamer}

\mode<presentation> {

\usetheme{Madrid}

% Burnt orange
\definecolor{burntorange}{rgb}{0.8, 0.33, 0.0}
\colorlet{beamer@blendedblue}{burntorange}
% Pale yellow
\definecolor{paleyellow}{rgb}{1.0, 1.0, 0.953}
\setbeamercolor{background canvas}{bg=paleyellow}
% Secondary and tertiary palett
\setbeamercolor*{palette secondary}{use=structure,fg=white,bg=burntorange!80!black}
\setbeamercolor*{palette tertiary}{use=structure,fg=white,bg=burntorange!60!black}

% To remove the footer line in all slides uncomment this line
%\setbeamertemplate{footline}
% To replace the footer line in all slides with a simple slide count uncomment this line
%\setbeamertemplate{footline}[page number]

% To remove the navigation symbols from the bottom of all slides uncomment this line
%\setbeamertemplate{navigation symbols}{}
}

\usepackage{amsmath}
\usepackage{bm}
\usepackage{breqn}
\usepackage{graphicx} % for figures
\usepackage{subcaption} % for subplots 
\usepackage[labelsep=space,tableposition=top]{caption}
\renewcommand{\figurename}{Fig.} 
\usepackage{cleveref}
\usepackage{caption,subcaption}% http://ctan.org/pkg/{caption,subcaption}
\usepackage{booktabs} % Allows the use of \toprule, \midrule and \bottomrule in tables


% To print 2 slides on a page
%\usepackage{handoutWithNotes}
%\pgfpagesuselayout{2 on 1}[border shrink=2mm]
%----------------------------------------------------------------------------------------
%	TITLE PAGE
%----------------------------------------------------------------------------------------
% The short title appears at the bottom of every slide, the full title is only on the title page
\title[CE394M: isoparametric - gauss integration]{CE394M: Isoparametric elements and Gauss integration} 
\author{Krishna Kumar} % name
\institute[UT Austin] % institution 
{
University of Texas at Austin \\
\medskip
\textit{
  \url{krishnak@utexas.edu}} % Your email address
}
\date{\today} % Date, can be changed to a custom date

\begin{document}

\begin{frame}
\titlepage % title page as the first slide
\end{frame}

\begin{frame}
 % Table of contents slide, comment this block out to remove it
 \frametitle{Overview}
 % Throughout your presentation, if you choose to use \section{} and \subsection{} 
 % commands, these %will automatically be printed on this slide as an overview 
 \tableofcontents
\end{frame}

%----------------------------------------------------------------------------------------
% slides
%----------------------------------------------------------------------------------------
\section{Rectangular elements}
%------------------------------------------------
\begin{frame}
\frametitle{4-noded rectangular element}
\begin{figure}[ht]
	\centering
	\includegraphics[width=0.85\textwidth]{figs/four-noded-quadrilateral.png}
	\caption*{Four-node rectangular element. The nodes are by definition numbered counter-clockwise.}
\end{figure}
\end{frame}

%------------------------------------------------
\begin{frame}
\frametitle{4-noded rectangular element}
As the element has four nodes, it is necessary to start with a polynomial that has four
parameters.
\mode<beamer>{
	\begin{equation*}
			T e = \alpha_0^e + \alpha_1^e x + \alpha_2^e y + \alpha_3^e xy
	\end{equation*}
	It is possible to express ($\alpha_0^e, \alpha_1^e, \alpha_2^e, \alpha_3^e$) in terms of the nodal values ($T_1^e, T_2^e, T_3^e , T_4^e$). A derivation shape functions is tedious as it is necessary to invert a 4 x 4 matrix.
	
	The Shape Functions should be 1 at each node, and 0 otherwise can be used to determine the 4 coefficients.
}
\mode<handout>{
	\vspace{5cm}
}
\end{frame}


%------------------------------------------------
\begin{frame}
\frametitle{4-noded rectangular element}
An alternative and more elegant approach is to construct the shape functions by the \textbf{tensor
product method.} This is based on taking products of one-dimensional shape functions.

\begin{figure}[ht]
	\centering
	\includegraphics[width=\textwidth]{figs/4noded-quad-tensor-product.png}
	\caption*{Construction of two dimensional shape functions.}
\end{figure}

\end{frame}


%------------------------------------------------
\begin{frame}
\frametitle{4-noded rectangular element}
For example, $N_2^e$, which has to have the value one at node 2 and zero at the other nodes,
is obtained by taking the product of the one-dimensional shape functions $N_2^{e,1d}(x)$ and 
$N_1^{e,1d}(y)$.

\mode<beamer>{
	\begin{equation*}
	N_2^e = N_2^{e,1d}(x) \times N_1^{e,1d}(y)
	\end{equation*}
	
	As visible in the figure above the product $ N_2^{e,1d}(x) \times N_1^{e,1d}(y)$ has the 
	value one at node 2 and is zero at nodes 1, 3 and 4.
	
	\begin{equation*}
	N_2^e (x, y) = \frac{x - x_1^e}{x_2^e - x_1^e}\frac{y - y_4^e}{y_1^e - y_4^e} = -\frac{1}{A^e}(x - x_1^e)(y - y_4^e)
	\end{equation*}
}
\mode<handout>{
	\vspace{5cm}
}
\end{frame}


%------------------------------------------------
\begin{frame}
\frametitle{4-noded rectangular element}

The four shape functions, also called \textbf{bilinear shape functions}, for the quadrilateral element are:
\begin{align*}
	N_1^e (x, y) & = \frac{x - x_2^e}{x_1^e - x_2^e}\frac{y - y_4^e}{y_1^e - y_4^e} = -\frac{1}{A^e}(x - x_2^e)(y - y_4^e) \\
	%
	N_2^e (x, y) & = \frac{x - x_1^e}{x_2^e - x_1^e}\frac{y - y_4^e}{y_1^e - y_4^e} = -\frac{1}{A^e}(x - x_1^e)(y - y_4^e) \\
	%
	N_3^e (x, y) & = \frac{x - x_1^e}{x_2^e - x_1^e}\frac{y - y_1^e}{y_4^e - y_1^e} = -\frac{1}{A^e}(x - x_1^e)(y - y_1^e) \\
	%
	N_4^e (x, y) & = \frac{x - x_2^e}{x_1^e - x_2^e}\frac{y - y_1^e}{y_4^e - y_1^e} = -\frac{1}{A^e}(x - x_2^e)(y - y_1^e)\\
\end{align*}

where $A^e$ is the area of the element.

\end{frame}


%------------------------------------------------
\begin{frame}
\frametitle{4-noded rectangular element}
The four shape functions are plotted in the following
figure:
\begin{figure}[ht]
	\centering
	\includegraphics[width=\textwidth]{figs/4noded-sf.png}
	\caption*{Four shape functions of the rectangular element (on $[0, 2] \times [0, 2]$).}
\end{figure}
\end{frame}


%------------------------------------------------
\begin{frame}
\frametitle{4-noded rectangular element}
If the sf equations are used for interpolating the temperature field over arbitrary quadrilaterals, the scalar field (e.g., temperature) across the element boundaries will be not continuous.
\begin{figure}[ht]
	\centering
	\includegraphics[width=\textwidth]{figs/discontinous-rectangle.png}
	\caption*{Two element mesh with rectangle shape functions. Although the two nodal values on the
		edge agree, the temperature distribution is still discontinuous.}
\end{figure}
\end{frame}

\note{
	The computed shape functions are suitable for rectangles and could be used with meshes consisting only of rectangles, but they are not suitable for arbitrary quadrilaterals.
	
	Therefore, these shape functions are of limited use for practical applications. To obtain
	the shape functions for arbitrary quadrilaterals we need to visit the idea of isoparametric
	mapping.
}

%------------------------------------------------
\section{Isoparametric elements}
%------------------------------------------------
\begin{frame}
\frametitle{Isoparametric mapping in 1D}
\mode<beamer>{
	\begin{figure}[ht]
		\centering
		\includegraphics[width=0.85\textwidth]{figs/1d-isoparametric.png}
	\end{figure}
}
\mode<handout>{
	\vspace{5cm}
}
\end{frame}
\note{Although isoparametric mapping is not particularly useful in one dimension, it is very	helpful for understanding the general approach.
}

%------------------------------------------------
\begin{frame}
\frametitle{Isoparametric mapping in 1D}
consider a coordinate transformation which transforms (maps) the coordinate $x$ into a local (element specific) coordinate $\xi$:	
\begin{figure}[ht]
	\centering
	\includegraphics[width=0.85\textwidth]{figs/1d-isoparametric-shapefn.png}
\end{figure}
The coordinate $\xi$ fulfills the relationships:

\mode<beamer>{
	$x = x^e_1$ at $\xi = -1$ and $x = x^e_2$ at $\xi=1$
}
\mode<handout>{
	\vspace{1cm}
}

``stretch transformation'' of $x(\xi)$: 

\mode<beamer>{
	\begin{align*}
		x(\xi) & = x_1^e + \frac{1}{2}(x_2^e - x_1^e)(1+\xi) \\
			& = \frac{x_1^e + x_2^e}{2} + \frac{x_2^e - x_1^e}{2}\xi
	\end{align*}
}
\mode<handout>{
	\vspace{1cm}
}
\end{frame}


%------------------------------------------------
\begin{frame}
\frametitle{Isoparametric mapping in 1D}
The shape functions $N_1^e = (1 - x/l)$ and $N_2^e = (x/l)$ can also be expressed using $\xi$:
\mode<beamer>{
	\begin{equation*}
		N_1^e(\xi) = \frac{1}{2}(1-\xi) \qquad N_2^e(\xi) = \frac{1}{2}(1+\xi)
	\end{equation*}
}
\mode<handout>{
	\vspace{1cm}
}

The key idea of the isoparametric concept is to use these shape functions for writing the
coordinate transformation between $x$ and $\xi$ 

\mode<beamer>{
	\begin{align*}
	x(\xi) & = N_1^e(\xi)x_1^e + N_2^e(\xi)x_2^e  \\
		   & = \frac{1}{2}(1-\xi) x_1^e + %
		    \frac{1}{2}(1+\xi) x_2^e \\
		   & = \frac{x_1^e + x_2^e}{2} + \frac{x_2^e - x_1^e}{2}\xi
	\end{align*}
}
\mode<handout>{
	\vspace{3cm}
}
\begin{figure}[ht]
	\centering
	\includegraphics[width=0.65\textwidth]{figs/1d-isoparametric-shapefn.png}
\end{figure}
\end{frame}


%------------------------------------------------
\begin{frame}
\frametitle{Isoparametric mapping in 1D}

\begin{itemize}
	\item The coordinate $\xi$ is usually called the \textbf{natural coordinate} and always lies by definition between -1 and +1.
	
	\item The parent element is solely for numerical purposes. 
	
	\item The finite element analysis is still performed over the physical domain.
\end{itemize}


In an isoparametric element the field variable, like displacement, is approximated with the
same set of shape functions as those used for the coordinate transformation:

\mode<beamer>{
	\begin{equation*}
	u(\xi) = N_1^e(\xi)a_1^e + N_2^e(\xi)a_2^e  
	\end{equation*}
}
\mode<handout>{
	\vspace{1cm}
}

To compute the derivatives which appear in the weak form the chain rule is used:

\mode<beamer>{
	\begin{equation*}
		\frac{du}{dx} = \frac{du}{d\xi}\frac{d\xi}{dx}
	\end{equation*}
	The derivative $d\xi/dx$ is determined from the mapping between $\xi$ and $x$
}
\mode<handout>{
	\vspace{1cm}
}
\end{frame}


%------------------------------------------------
\section{Isoparametric quadrilateral elements}
%------------------------------------------------
\begin{frame}
\frametitle{Isoparametric mapping of a quadrilateral element}
The idea of isoparametric mapping is used for deriving shape functions for arbitrary quadrilateral elements:
\begin{figure}[ht]
	\centering
	\includegraphics[width=\textwidth]{figs/2d-isoparametric-shapefn.png}
\end{figure}
\end{frame}

%------------------------------------------------
\begin{frame}
\frametitle{Isoparametric mapping of a quadrilateral element}
The bi-unit square is the parent domain and $\xi$ and $\eta$ are its natural coordinates. 

To map points from the parent domain onto the quadrilateral in the physical domain the four nodal
shape functions are used:

\mode<beamer>{
	\begin{equation*}
	x(\xi, \eta) = \mathbf{N}^{4Q}(\xi, \eta)x^e \quad
	y(\xi, \eta) = \mathbf{N}^{4Q}(\xi, \eta)y^e
	\end{equation*}
	where $N^{4Q}(\xi, \eta)$ are the four-node element shape functions in the natural coordinates and $x^e$ and $y^e$ are the vectors of the element coordinates:
	\begin{equation*}
	\mathbf{x}^e = 
	\begin{bmatrix} 
	x_1^e \\
	x_2^e \\
	x_3^e \\
	x_4^e \\
	\end{bmatrix}
	\quad
	\mathbf{y}^e = 
	\begin{bmatrix} 
	y_1^e \\
	y_2^e \\
	y_3^e \\
	y_4^e \\
	\end{bmatrix}
	\end{equation*}
	
}
\mode<handout>{
	\vspace{5cm}
}
\end{frame}

%------------------------------------------------
\begin{frame}
\frametitle{Isoparametric mapping of a quadrilateral element}
As the parent element is a bi-unit square its shape functions are identical to those of the
rectangular element expressed in $\xi$ and $\eta$ coordinates.
\begin{align*}
	N_1^{4Q}(\xi, \eta) & = \frac{1}{4}(1-\xi)(1-\eta)\\
	N_2^{4Q}(\xi, \eta) & = \frac{1}{4}(1+\xi)(1-\eta)\\
	N_3^{4Q}(\xi, \eta) & = \frac{1}{4}(1+\xi)(1+\eta)\\
	N_4^{4Q}(\xi, \eta) & = \frac{1}{4}(1-\xi)(1+\eta)\\
\end{align*}
\end{frame}


%------------------------------------------------
\begin{frame}
\frametitle{Isoparametric shape functions}
The temperature will be approximated with the same shape functions:

\mode<beamer>{
	\begin{equation*}
		T^e = \mathbf{N}^{4Q}(\xi, \eta) \mathbf{a}^e
	\end{equation*}
}
\mode<handout>{
	\vspace{1cm}
}
The element is called \textit{isoparametric} because the temperature approximation and the mapping of the geometry is accomplished with the same shape functions.

The displacement will be approximated as:
\mode<beamer>{
	\begin{align*}
	\mathbf{u}^e & = \mathbf{N}^{4Q}(\xi, \eta) \mathbf{a}^e \\
	& = \begin{bmatrix}
	N_1^{4Q} & 0 & N_2^{4Q} & 0 & N_3^{4Q} & 0 & N_4^{4Q} & 0 \\
	0 & N_1^{4Q} & 0 & N_2^{4Q} & 0 & N_3^{4Q} & 0 & N_4^{4Q} \\
	\end{bmatrix}
	\begin{bmatrix}
		a_{1x}^e \\
		a_{1y}^e \\
		a_{2x}^e \\
		a_{2y}^e \\
		a_{3x}^e \\
		a_{3y}^e \\
		a_{4x}^e \\
		a_{4y}^e \\
	\end{bmatrix}
	\end{align*}
}
\mode<handout>{
	\vspace{3cm}
}
\end{frame}

%------------------------------------------------
\begin{frame}
\frametitle{Derivatives isoparametric shape functions}
The gradient of the temperature for the four-node (isoparametric) quadrilateral element is:

\mode<beamer>{
	\begin{equation*}
	\nabla T = \mathbf{B}^{e}\mathbf{a}^e
	\end{equation*}
}
\mode<handout>{
	\vspace{1cm}
}
with
\mode<beamer>{
	\begin{equation*}
	\mathbf{B}^e = %
	\begin{bmatrix}
	\frac{\partial N_1^{4Q}}{\partial x} &
	\frac{\partial N_2^{4Q}}{\partial x} &
	\frac{\partial N_3^{4Q}}{\partial x} &
	\frac{\partial N_4^{4Q}}{\partial x} \\
	\frac{\partial N_1^{4Q}}{\partial y} &
	\frac{\partial N_2^{4Q}}{\partial y} &
	\frac{\partial N_3^{4Q}}{\partial y} &
	\frac{\partial N_4^{4Q}}{\partial y} \\	
	\end{bmatrix}
	\end{equation*}
}
\mode<handout>{
	\vspace{3cm}
}
\end{frame}

%------------------------------------------------
\begin{frame}
\frametitle{Derivatives isoparametric shape functions}
To compute shape function derivatives the chain rule will be used:

\mode<beamer>{
	\begin{align*}
		\frac{\partial N_I^{4Q}}{\partial \xi} & = \frac{\partial N_I^{4Q}}{\partial x} \frac{\partial x}{\partial \xi} + %
		\frac{\partial N_I^{4Q}}{\partial y} \frac{\partial y}{\partial \xi} \\
	%
		\frac{\partial N_I^{4Q}}{\partial \eta} & = \frac{\partial N_I^{4Q}}{\partial x} \frac{\partial x}{\partial \eta} + %
		\frac{\partial N_I^{4Q}}{\partial y} \frac{\partial y}{\partial \eta} \\
	\end{align*}
}
\mode<handout>{
	\vspace{2cm}
}
written as matrices and vectors this becomes:
\mode<beamer>{
	\begin{equation*}
	\begin{bmatrix}
		\frac{\partial N_I^{4Q}}{\partial \xi} \\
		\frac{\partial N_I^{4Q}}{\partial \eta} \\
	\end{bmatrix}
	=% 
	\begin{bmatrix}
		\frac{\partial x}{\partial \xi} & \frac{\partial y}{\partial \xi} \\
		\frac{\partial x}{\partial \eta} & \frac{\partial y}{\partial \eta} \\
	\end{bmatrix}
	\begin{bmatrix}
		\frac{\partial N_I^{4Q}}{\partial x} \\
		\frac{\partial N_I^{4Q}}{\partial y} \\
	\end{bmatrix}
	\end{equation*}
}
$\mathbf{J}^e$ is the Jacobian which contains the derivatives of the physical coordinates with respect
to the natural coordinates.
\mode<handout>{
	\vspace{3cm}
}
\end{frame}

%------------------------------------------------
\begin{frame}
\frametitle{Derivatives isoparametric shape functions}
The derivatives required for the weak form are computed by
inverting the above expression:
\mode<beamer>{
	\begin{equation*}
	\begin{bmatrix}
		\frac{\partial N_I^{4Q}}{\partial x} \\
		\frac{\partial N_I^{4Q}}{\partial y} \\
	\end{bmatrix}
	= %
	(\mathbf{J}^e)^{-1}
	\begin{bmatrix}
	\frac{\partial N_I^{4Q}}{\partial \xi} \\
	\frac{\partial N_I^{4Q}}{\partial \eta} \\
	\end{bmatrix}
	\end{equation*}
}
\mode<handout>{
	\vspace{2.5cm}
}
The inverse of $\mathbf{J}^e$ is:
\mode<beamer>{
	\begin{equation*}
	{\mathbf{J}^e}^{-1}
	= \frac{1}{\left| \mathbf{J}^e\right|}
	\begin{bmatrix}
		\frac{\partial y}{\partial \eta} & -\frac{\partial y}{\partial \xi} \\
		-\frac{\partial x}{\partial \eta} & \frac{\partial y}{\partial \xi} \\
	\end{bmatrix}
	\end{equation*}
}
\mode<handout>{
	\vspace{2.5cm}
}
Where $\left|\mathbf{J}^e\right|$ is the determinant of the Jacobian, which represents the ratio of an area element in the physical domain to the corresponding area element in the parent domain.
\end{frame}


%------------------------------------------------
\begin{frame}
\frametitle{Derivatives isoparametric shape functions}
The isoparametric mapping is used to compute the Jacobian:
\mode<beamer>{
	\begin{equation*}
	x(\xi, \eta) = \mathbf{N}^{4Q}(\xi, \eta)\mathbf{x}^e \quad
	y(\xi, \eta) = \mathbf{N}^{4Q}(\xi, \eta)\mathbf{y}^e \quad
	\end{equation*}
}
\mode<handout>{
	\vspace{3cm}
}
which leads to:
\begin{equation*}
	\mathbf{J}^e = %
	\begin{bmatrix}
		\frac{\partial N_1^{4Q}}{\partial \xi} &
		\frac{\partial N_2^{4Q}}{\partial \xi} &
		\frac{\partial N_3^{4Q}}{\partial \xi} &
		\frac{\partial N_4^{4Q}}{\partial \xi} \\
		\frac{\partial N_1^{4Q}}{\partial \eta} &
		\frac{\partial N_2^{4Q}}{\partial \eta} &
		\frac{\partial N_3^{4Q}}{\partial \eta} &
		\frac{\partial N_4^{4Q}}{\partial \eta} \\	
	\end{bmatrix}
	\begin{bmatrix}
		x_1^e & y_1^e\\
		x_2^e & y_2^e\\
		x_3^e & y_3^e\\
		x_4^e & y_4^e\\
	\end{bmatrix}
\end{equation*}
\end{frame}

%------------------------------------------------
\begin{frame}
\frametitle{Higher order quadrilateral element}

\begin{figure}[ht]
	\centering
	\includegraphics[width=0.85\textwidth]{figs/9noded-quadrilateral.png}
	\caption*{Nine-node ispoarametric quadrilateral in parameter (left) and physical space (right).}
\end{figure}
\end{frame}
\note{
	Higher order quadrilateral elements provide the ability to model curved edges. The advantage of curved edges is that fewer elements can be used around holes and other curved surfaces than with straight-sided elements. \\
	
	The nine-node isoparametric element is constructed as a tensor product of the one-dimensional quadratic shape functions. The $\mathbf{B}_e$ matrix for the nine noded element is computed with the same approach as discussed in the previous section.
}

\section{Effect of element shape}
%------------------------------------------------
\begin{frame}
\frametitle{Effect of element shape on isoparametric mapping}
Consider the following isoparametric mapping for a four-node quadrilateral element:
\begin{figure}[ht]
	\centering
	\includegraphics[width=\textwidth]{figs/2d-isoparametric-example.png}
\end{figure}
\end{frame}

%------------------------------------------------
\begin{frame}
\frametitle{Effect of element shape on isoparametric mapping}
The Jacobian of this mapping:
\begin{equation*}
	\mathbf{J}^e%
	=% 
	\begin{bmatrix}
		\frac{\partial x}{\partial \xi} & \frac{\partial y}{\partial \xi} \\
		\frac{\partial x}{\partial \eta} & \frac{\partial y}{\partial \eta} \\
	\end{bmatrix}
\end{equation*}
is computed from
\mode<beamer>{
	\begin{align*}
		x & = 0.0 N_1^{4Q} + 5.0  N_2^{4Q} + 3.0  N_3^{4Q} + 0.0  N_4^{4Q} \\
		% 
		& = 2 \xi - \frac{1}{2}\eta - \frac{1}{2}\xi \eta + 2 \\
		%
		\vspace{1em}
		%
		x & = 0.0 N_1^{4Q} + 0.0  N_2^{4Q} + 3.0  N_3^{4Q} + 5.0  N_4^{4Q} \\
		% 
		& = - \frac{1}{2} \xi + 2 \eta - \frac{1}{2}\xi \eta + 2 \\
	\end{align*}
}
\mode<handout>{
	\vspace{4cm}
}
\end{frame}


%------------------------------------------------
\begin{frame}
\frametitle{Effect of element shape on isoparametric mapping}
After some algebra we obtain:
\begin{equation*}
\mathbf{J}^e%
=% 
\begin{bmatrix}
2 - \frac{\eta}{2} & -\frac{1}{2} -\frac{1}{2}\eta \\
 -\frac{1}{2} -\frac{1}{2}\xi & 2 - \frac{1}{2}\xi \\
\end{bmatrix}
\end{equation*}

The Jacobian has to be invertible for computing the derivatives of the shape functions with
respect to the physical coordinates.

For the mapping to be invertible, the determinant of the Jacobian has to be larger than zero over the entire element:

\mode<beamer>{
	\begin{equation*}
		det \mathbf{J}^e = \frac{5}{4}(3 - \xi - eta) > 0
	\end{equation*}
}
\mode<handout>{
	\begin{equation*}
	det \mathbf{J}^e = \frac{5}{4}(3 - \xi - eta)
	\end{equation*}
}
which is the case for this mapping.
\end{frame}


%------------------------------------------------
\begin{frame}
\frametitle{Effect of element shape on isoparametric mapping}
In contrast to the previous mapping, it can be for the following mapping shown:
\begin{figure}[ht]
	\centering
	\includegraphics[width=\textwidth]{figs/2d-isoparametric-example-negative-jacobian.png}
\end{figure}
that the determinant of the Jacobian is zero or negative close to the non-convex corner.
\end{frame}

\note{
	Notice that some region of the parent element close to node 3 is mapped outside the
	physical domain. If such non-convex elements are present in the finite element mesh, the
	results of the finite element computation will be useless.
}
\end{document}