\documentclass[a4paper,12pt]{article}
\usepackage{graphicx}
\usepackage[left=30mm, right=30mm, top=30mm, bottom=30mm]{geometry}
\usepackage{amsmath}
\usepackage{siunitx}
\usepackage{fancyhdr}
\usepackage{url}
\pagestyle{fancy}
%-------------------------------------------------------------------------------
\lhead{\textbf{CE394M}}
\rhead{\textbf{Advanced Analysis in Geotechnical Engineering}}
\cfoot{\thepage}
%-------------------------------------------------------------------------------

\begin{document}
\begin{centering}
	\textbf{
		Assignment 9: Cam-Clay\\
	}
\end{centering}

\vspace{1em}
\section{Simple shear}

\begin{enumerate}
	
	\item Kaolin is reconstituted to a slurry and is then permitted to re-consolidate one-dimensionally. It eventually reaches a vertical effective stress of 100 kPa in a Simple Shear Apparatus:
	\begin{enumerate}
		\item Estimate its water content
		\item Predict its undrained shear strength
		\item It is then permitted to drain while it continues shearing; predict its drained shear strength.
		\item What volumetric change will the sample eventually suffer during its drained shearing? How would this estimate be changed if a pre-consolidation stress of 1000 kPa had first been imposed during the initial setting-up.
	\end{enumerate}

	\item London Clay is normally consolidated to 1000 kPa and then permitted to swell back into equilibrium (with zero pore pressure) under a normal stress of 50 kPa
	\begin{enumerate}
		\item Predict both drained and undrained shear strengths using SSA Cam Clay.
		\item Estimate the pore water pressure consistent with your estimate of the undrained shear strength. Comment on the magnitude in relation to the probable behavior of heavily over-consolidated London Clay exposed in an excavation.
	\end{enumerate}
\end{enumerate}

\section{Triaxial tests}
\begin{enumerate}
	\item Establish expression for TX compression Cam Clay parameter $M$ as a function of $\phi_{crit}$. Assume that the test eventually come to mobilize $\phi_{crit}$ in the vertical plane. Hint: Use earth pressure co-efficient to relate different component of stresses. 
	
	\item A saturated clay is characterized by these Cam-Clay parameters: $M = 0.87$, $\lambda = 0.091$, $\kappa = 0.035$ and $\Gamma = 2.072$ at $p^\prime = 1 kPa$. 
	Consider two different soil specimens consolidated to the same $p_c^\prime = 100 kPa$. Specimen A is isotropically consolidated to $p_0^\prime = 100 kPa$, while Specimen B is anisotropically consolidated to $p_0^\prime = 100 kPa$ with $K_c = \sigma_{1c}^\prime / \sigma_{3c}^\prime = 2.0$. 
	\begin{enumerate}
		\item Sketch the initial states, paths and yield surfaces for each specimen in $q-p^\prime$ and $v - \ln p^\prime$ space. 
		\item Use the MCC model to predict the undrained shear strength fro the two speciments. Compare your results.
		\item Use the MCC model to predict the drained $q_f$ at failure.
	\end{enumerate}

	\item Weald clay is reconstituted as a saturated slurry and isotropically consolidated to $p^\prime = 100kPa$, before being allowed to swell back to 70kPa. Use OCC.
	\begin{enumerate}
		\item What will be its water content?
		\item It is then to be subjected to undrained triaxial compression. At what deviatoric stress $q$ might the sample yield? Estimate the axial strain at yield (assuming effective Poisson's ratio of 0.15).
		\item If $q$ is allowed to increase a further 10\% as the undrained test progresses, search for a consistent value of the mean effective stress $p^\prime$ at that stage.
		\item What ultimate undrained strength $q_u$ should be recorded?
		\item What volumetric strain should occur if the sample were finally allowed to drain while
		shearing continued, and what would be the ultimate strength?
	\end{enumerate}
\end{enumerate}

\section{Material parameters and constitutive law}	
	\begin{enumerate}
	\item Develop Modified Cam-Clay parameters for Young San Francisco Bay Mud from the triaxial compression data and consolidation test data provided. Use $\kappa = 0.2 \lambda$
	
	\item Develop the stiffness matrix in the cam-clay.ipynb file provided. Simulate two undrained triaxial tests using the parameters derived in the previous problem for the Modified Cam-Clay model. In addition to parameters determined above, use $N = 4$  and $\nu = 0.2$
	
	\begin{enumerate}
		\item Isotopically consolidated undrained compression test of normally consolidated clay.
		Normally consolidated to 100 kPa and then sheared in undrained conditions. For the undrained shear part of the test:
		\begin{enumerate}
			\item Determine the initial void ratio before shearing.
			\item Plot deviatoric stress $q$ versus axial strain $\varepsilon_a$
			\item Plot the stress path in $q-p^\prime$ plane and the state path in $e - \ln p^\prime$ plane
			\item Plot excess pore pressure $u$ versus axial strain
		\end{enumerate}
		
		\item Isotopically consolidated undrained compression test of overconsolidated clay.
		Normally consolidated to 450 kPa isotopically, unloaded isotropically to 100 kPa
		and then sheared in undrained conditions. For the undrained shear part of the test:
		\begin{enumerate}
			\item Determine the initial void ratio before shearing.
			\item Plot deviatoric stress $q$ versus axial strain $\varepsilon_a$
			\item Plot the stress path in $q-p^\prime$ plane and the state path in $e - \ln p^\prime$ plane
			\item Plot excess pore pressure $u$ versus axial strain
		\end{enumerate}
		Note: For the overconsolidated condition, use the elasto-plastic D matrix after the stress state reaches the yield surface. Use the elastic D matrix before yielding.
	\end{enumerate}
	\end{enumerate}
\begin{figure}[!h]
	\includegraphics[width=\textwidth]{figs/soil-data-sheet.png}
\end{figure}
\end{document}

